\documentclass{article}

%Packages
\usepackage{graphicx}
\usepackage{grffile}
\usepackage{float}

%Margins
\usepackage[
	margin=2cm,
	includefoot
	]{geometry}

\graphicspath{{images/}}

%Headers and Footers
\usepackage{fancyhdr}
\pagestyle{fancy}
\fancyhead{}
\fancyfoot{}
\fancyfoot[R]{\thepage}
\renewcommand{\headrulewidth}{0pt}
\renewcommand{\footrulewidth}{0pt}

%Details
\title{
	Architecture Specification
}
\author{Illusion Solutions}

%Document start
\begin{document}
	
	%Title Page
	\newgeometry{top=4.5cm}
	\begin{titlepage}
		\begin{center}
			\line(1,0){300} \\
			[0.1cm]
			\textsc{\Huge
				PowerCloud\\
				Architecture Specification
			} \\
			\textsc{\large version 2.0}\\
			[0.1cm]
			\line(1,0){300} \\
			[2.0cm]
			\textsc{\Large
				Illusion Solutions
			} \\
			[3.5cm]
			
		\end{center}
		\begin{flushright}
			\textsc{\Large
				Stuart Andrews\\ 
				12153983\\
				Marc Antel\\
				12026973\\
				Mothusi Masibi\\
				12004589\\
				Brandon Wardley\\
				29005150\\
				[4.0cm]
			}
		\end{flushright}
		\begin{center}
			\today
		\end{center}
	\end{titlepage}

\newpage
\restoregeometry
\tableofcontents
\thispagestyle{empty}

\newpage

\section{Introduction}

The purpose of this document is to describe the Architectural 
Specifications of the PowerCloud project. This document is an ongoing 
piece of work, and is updated as the project progresses. The 
architectural scope, technologies, access and integration channels will 
be discussed, and testable quality requirements will be provided where 
applicable.

\section{Architecture requirements}

This section contains a brief discussion of the software requirements of 
the PowerCloud system. It includes the architectural scope, the quality 
requirements, the integration requirements, access channel requirements, 
and architectural constraints.

	\subsection{Architectural scope}
	
	The software architecture used within the PowerCloud system must 
	satisfy certain requirements brought forward by the client. These 
	requirements need to be satisfied in order to ensure the system 
	provides a satisfactory level of performance. The requirements are as 
	follows:

	\begin{itemize}
		\item The system must provide a reliable persistence service 
		which stores data in a readable format. This data must be easily 
		maintainable.
		\item The system must be secure. Only authorised users should be 
		able to access and use the system. This includes both human 
		operators and Particle devices. Data sent within the system must 
		be secure, and should not be accessible by outside parties.
		\item The system must provide reporting functionality. Stored 
		data must be represented to the user in an understandable manner. 
		This representation must be useful to the user, and provide 
		additional understanding of the data as a whole, via the use of 
		valid statistics.
		\item The system must provide reliability, such that the data 
		sent and retrieved is both accurate and consistent. The system, 
		and devices within the system, must recover gracefully if a 
		failure does occur. If a part of the system fails, data should 
		not be lost.
		\item The system must be able to cater to a large number of 
		users, and no user should experience a noticeable lack in 
		performance due to contention. This includes both human operators 
		and Particle devices.
		\item The system should be easy to use, such that operators can 
		easily access the required data, and that devices within the 
		system be easy to configure and use.
		\item The system must provide compatibility, such that the data 
		can be accessed from any computer which meets the minimum 
		requirements.
	\end{itemize}
	
	\newpage

	\subsection{Quality requirements}
	
	The following quality requirements were established, and were found 
	to be relevant to the PowerCloud system. These requirements should be 
	quantifiable and testable, and are listed in order of priority.
	
	\begin{enumerate}
		\item Recoverability
			\begin{itemize}
				\item If a device goes offline, data should not lost for 
				a given period, determined by the frequency at which 
				measurements are taken.
				\item If a server were to cease operation, a replacement 
				must be provided as soon as possible, in order to 
				minimize impact on user operations.
			\end{itemize}
		\item Quality
			\begin{itemize}
				\item The server needs to ensure that data is in a valid 
				format.
				\item The server needs to ensure that data is consistent, 
				and if any deviations occur, the operators must be 
				notified.
			\end{itemize}
		\item Reliability
			\begin{itemize}
				\item Data measured by the devices must be accurate, up 
				to a given level specified by the operator.
				\item The server must ensure a 99.8\% uptime, outside of 
				regular maintenance periods.
			\end{itemize}
		\item Security
			\begin{itemize}
				\item Device communication must be encrypted to avoid 
				interception of sensitive data.
				\item Unauthorized operators and devices must not be able 
				to access the system.
			\end{itemize}
		\item Usability
			\begin{itemize}
				\item The PowerCloud web interface must be intuitive and 
				easy to use.
				\item Feedback must be provided to users after actions 
				are performed, informing them as to the result of the 
				action.
				\item If an error should occur, the user must be made 
				aware of the error.
				\item Operation guidelines must be provided, to ensure 
				the user can properly operate the system.
			\end{itemize}
		\item Scalability
			\begin{itemize}
				\item The server needs to handle up to 100 000 
				simultaneous connections from deployed devices.
			\end{itemize}
		\item Maintainability
			\begin{itemize}
				\item Any upgrades to the system must not change the 
				contents in the database.
				\item A module within the system can be changed without 
				affecting any of the other modules.
			\end{itemize}
	\end{enumerate}
	
	\newpage
	
	\subsection{Integration requirements}
	
	The first stage in the PowerCloud system is the measuring device itself. The device consists out of a number of components. The first is a power meter, responsible for taking readings from 
	the connected appliance and sending these measurements on to the Photon. This is achieved by sending a pulse to an input pin on the Photon board, where every high pulse indicates a unit 
	measured.\\
	
	The second device, or the Photon, sends the readings to the 
	application server by using MQ Telemetry Transport (MQTT), which is a 
	publish-subscribe messaging protocol which is used on top of the 
	TCP/IP protocol. Using MQTT, the Photon publishes data to the 
	application server in a JavaScript Object Notation (JSON) format. The 
	application server listens for any incoming MQTT messages, and 
	processes those which it receives.\\
	
	Additionally, the Photon also controls a relay, which controls 
	whether the connected is one or off, by controlling the flow of 
	current to the device. This is achieved by using one of the Photon 
	pins as an output, which connects the the relay trigger. A high 
	signal indicates that the relay is on, and the device has power, 
	whilst a low signal indicates that the relay is off, and the device 
	has no power. The relay can also be controlled via the web 
	application, allowing the user to turn it on and off, via signals 
	sent to the Photon device.\\
	
	The application server communicates with a Firebase persistence 
	service using a Representational State Transfer (REsT) architecture, 
	and as such communication uses the Hypertext Transfer Protocol 
	(HTTP). This means that communication will rely on using HTTP verbs, 
	such as POST and GET. The POST command is mainly used to send data to 
	the Firebase service, such as when data needs to be stored in the 
	Firebase database. The GET command is used to request data from the 
	Firebase service, and is used when previously stored data needs to be 
	retrieved from the database.\\
	
	In order for operators to access the data which has been stored on 
	the Firebase database, a web interface must be provided. For this 
	reason, a web server is used to connect to Firebase and retrieve any 
	needed data from the database. The web server will communicate with 
	Firebase in the same manner as the application server, using HTTP. 
	The web server will then process this data, and display it to the 
	operators via a web page.\\
	
	It is important to take note that each of these systems have certain 
	quality requirements which they must provide for. Reliable 
	communication must be provided, which will ensure that data is not 
	lost in transmission, or otherwise altered. Security must be provided 
	when transporting data over the public domain, to ensure that said 
	data is not accessed by unauthorized entities.
	
	\newpage
	
	\subsection{Access channel requirements}
	
	Users, in the form of operators, will need to access the system to 
	perform a number of tasks. The primary task will be to interact with 
	and access data stored on the database. Additional tasks may include 
	adding new devices to the system, or changing information relating to 
	devices already on the system. In order to perform these tasks, a web 
	application must be provided. This web application is provided by the 
	web server. It is necessary that the web application must be easy to 
	use and navigate, and must clearly display any and all required 
	information. The web application must also provide feedback to users 
	in a timely manner, such that they are made aware of the results of 
	their actions, or of any alerts or notifications that may require 
	their attention.\\
	
	The web application must be compatible with modern web browsers. As 
	the web application will primarily be used to interact with the 
	Firebase database, a working Internet connection will be required for 
	it to operate. The web application will be based on the REsT 
	architecture, and will communicate using the HTTP protocol.\\
	
	Another access channel that must be mentioned is the interaction 
	between the device and the appliance which it is measuring. An 
	electrical lead is used to attach the appliance to the device, as 
	well as to power it from a power source. In order for the correct 
	operations to occur, and to ensure the safety of both the device and 
	nearby operators, it is required that the appliance plug and source 
	both be correctly wired, as an incorrectly wired plug can cause 
	damage to both the device and potential operators.
	
	\subsection{Architectural constraints}
	
	The only major constraint in the PowerCloud system is that it 
	requires the use of the Particle device, through which readings are 
	taken and transmitted. This enforces the use of the following:\\
	
	\subsubsection{C++}
	The Particle device requires that the firmware be coded in C++.\\
	
	\subsubsection{Particle.io}
	The Particle device requires the use of the Particle.io API to access 
	much of the software and hardware functionality of the Photon board.

\newpage

\section{Architectural patterns and styles}

	\subsection{The Layering Pattern}
	
	The primary pattern used within the PowerCloud system is the Layering 
	Pattern. This pattern provides a number of benefits which encouraged its 
	use. Firstly, due to the manner in which it is structured, the layering 
	pattern separates different parts of the system implementation into 
	layers. Each of these layers can easily be removed or replaced without 
	effecting the implementation of other layers. This allows a great deal of 
	modularity, as one layer can be swapped out for another, equivalent 
	layer, allowing front- and back-end implementations to be changed as 
	needed.\\
	
	A second advantage is in the manner in which the layering pattern handles 
	communication between layers. Modules within a layer can only access the 
	modules which are in the same layer, or the next layer below itself. In 
	this way, communication between modules is restricted, such that the 
	client layers would not be able to directly communicate with the 
	persistence layers, which provides an added level of control, as well as 
	a reduction in complexity in the communication between different 
	modules.\\
	
	A third advantage provided by the layering pattern is that it allows each 
	layer and module to be developed independently of one another, which is 
	of great use in a software development environment, as different 
	developers are able to focus their efforts on and progress in the 
	development of separate modules without relying on the progress of other 
	modules, substituting their inputs and output with mock objects where 
	necessary.\\
	
	\noindent
	The layers used within the PowerCloud system are as follows:\\
	
		\begin{enumerate}
			\item Client Interface Layer\\
			The client interface layer contains the Particle device and web 
			application. It serves as the point of entry into the system for 
			users, represented by the Particle devices and human operators. 
			The Particle devices communicate with the application server by 
			using MQTT, whilst the human operators communicate operate with 
			the web server via a web application, which uses HTTP.
			\item Business Layer\\
			The business layer is where data and requests received from the 
			client interface layer are processed and responded to. It 
			contains the application server, which communicates with the 
			Particle devices, and the web server, which communicates with the 
			human operators via the web application. The servers interact 
			with the database mapping layer by passing storage and request 
			objects, which serve as containers for the data to be transferred.
			\item Database Mapping Layer
			The database mapping layer is responsible for interfacing between 
			the above business layer and the persistence layer below. It does 
			this by converting requests from the business layer into requests 
			which the persistence layer can understand and respond to. It 
			contains the persistence handler module.
			\item Persistence Layer\\
			The persistence layer is responsible for completing requests from 
			the database mapping layer. These requests can be in the form of 
			either storing data into the database, or retrieving previously 
			stored data. This layer currently contains the Firebase handler 
			module. The persistence layer receives objects to be stored or 
			retrieved from the database mapping layer, and then communicates 
			the database, using HTTP in the case of Firebase.
		\end{enumerate}

\newpage

\section{Architectural tactics and strategies}

The following tactics are being used, or are planned to be implemented. Additional 
tactics will need to be added to the project as further issues are identified.

	\begin{itemize}
		\item \textbf{Testing Framework}\\
		Testing Frameworks provide continuous testing of code via unit testing, 
		in order to reduce risks of system failure, provide rapid feedback on 
		developed units, and improve the maintainability of the code.\\
		
		Within the PowerCloud system, code is tested using the JUnit 
		framework, Jasmine framework and KarmaJS runner. The JUnit tests 
		are integrated into the Maven build, whilst the KarmaJS tests 
		will be integrated into Gulp.
		
		\item \textbf{Error/exception communication}\\
		Error and exception communication allows faults to be caught and 
		communicated throughout the system, such that they can be handled and 
		recovered from.\\
		
		Within the PowerCloud system, potential problems are caught using 
		either exceptions, which can be thrown and handled, or by using testing 
		methods which return a value which indicates that an error or 
		undesirable situation has occurred, alerting the system and allowing 
		the required actions to be taken.
		
		\item \textbf{Message integrity}\\
		Message integrity ensures that messages received within the system are 
		in the correct format, and attempts to determine whether the contents 
		within the messages have been changed in any way after being sent.\\
		
		Within the PowerCloud system, messages sent from the Particle device 
		are checked once they have been received by the application server. At 
		present, the checks ensure that the messages are in the correct JSON 
		format. Once greater knowledge has been established of the measured 
		appliance, the operators can set specific thresholds to define valid 
		data.
		
		\item \textbf{Passive redundancy}\\
		Passive redundancy can refer to redundancy systems which are set into 
		place to recover from potential faults and failures. This can refer to 
		both physical faults as well as data failures.\\
		
		The PowerCloud system makes use of the Firebase service for cloud 
		storage purposes. The Firebase service provides redundancy services for 
		their database services, ensuring that data is available at all times.

		\item \textbf{Authentication}\\
		Authentication refers to the security practise whereby attempts to gain 
		access to secure resources or services requires the possession of both 
		valid identification and the correct secret knowledge, such as a 
		password.\\
		
		Within the PowerCloud system, only users with valid authentication 
		credentials are able to access the data stored in the database. Users 
		are required to log in to the web application before they are able to 
		make use of its services.
		
		\item \textbf{Authorization}\\
		Authorization refers to the security practise whereby an entity must be 
		allowed access to resources based on whether they have permissions to 
		access said resources.\\
		
		The PowerCloud system only allows authorized Particle devices to store 
		information in the database, and user access privileges are determined 
		based on authorisation privileges assigned on a per user basis via 
		Firebase.
	\end{itemize}

\newpage

\section{Use of reference architectures}

	One of the major facets of the PowerCloud system is that it is an 
	Internet of Things (IoT) project. Being a relatively new development 
	within the computing world, there is a shortage of mature, well 
	defined architectures which centre around IoT, compared to other 
	branches of software development. Reference architectures are, 
	however, present and maturing, as the focus on and interest 
	surrounding IoT development continues to grow.
	
	\subsection{Reference IoT Layered Architecture}
	
		The Reference IoT Layered Architecture (RILA) is one such 
		developing architecture for use within IoT systems. Based on 
		high-level architectures which had previously been developed by 
		the Internet of Things Architecture (IoT-A) project, the RILA 
		architecture seeks to simplify abstract high-level architectures 
		into a form that is easier to understand and apply within a 
		system. The RILA architecture separates a system into a number of 
		layers, each of which only have access to the information within 
		their own layer or the information directly below it. Each of 
		these layers represent a part of the IoT system, and separate and 
		contain the various responsibilities within the system. The 
		layers are as follows.
		
		\begin{enumerate}
			\item Application Integration\\
			The application integration layer represents the channel 
			whereby the user has access to the IoT system. This includes 
			any applications whereby the user can access the system, such 
			as user interfaces, web services, or more complex 
			applications.
			\item Thing Integration\\
			The thing integration layer is responsible for the detection 
			of other IoT devices and services, and integrating them with 
			one another. This is achieved by establishing whether the 
			devices can possibly communicate with one another, and, if 
			possible, the sharing of scenarios and attached goals between 
			devices where applicable.
			\item Context Management\\
			The context management layer is responsible for containing 
			the business logic within the system. It controls the defined 
			goals of attached devices, as well as the various known 
			scenarios in which they operate and how they can achieve 
			those goals within those scenarios.
			\item Data Management\\
			The data management layer represents the persistence 
			services, or database, which is responsible for storing all 
			the data of attached devices.
			\item Device Management\\
			The device management layer is responsible for controlling 
			the registration of connected devices, as well as accepting 
			sensor measurements from the device integration layer. 
			Additionally, it is responsible for conveying status changes 
			in attached components received from the layers above itself 
			to the integration layer below it.
			\item Device Integration\\
			The device integration layer is responsible for providing 
			interconnectivity between between different device types, as 
			well as handling the measurements of the attached devices. It 
			functions as a translator, ensuring that a wide variety of 
			devices and attached components can successfully communicate 
			within a system.
		\end{enumerate}
		
	The reason why the RILA architecture was chosen is due to the fact 
	that it resembles the layered pattern which was initially chosen for 
	the PowerCloud system. It provides a number of solutions to problems 
	one might expect within an IoT system, and the separation of 
	responsibilities in different layers would allow branched development 
	within the team, without relying on the development of other layers. 
	Because the RILA architecture is designed for use in large IoT 
	systems with many difference devices, some of the layers and provided 
	functions are not suitable for use, or required, within the 
	PowerCloud system, at this point in its development.

\section{Access and integration channels}
	
	As mentioned earlier, the PowerCloud system has two main access 
	channels. The first access channel is in the form of a web 
	application, whereby human users are able to interact with the 
	system. The main forms of interaction for human users are the 
	retrieval of information from the database by requesting various 
	reports, the addition of new information into the database, namely 
	the addition of new devices, and editing already stored information, 
	namely the changing of a devices ID or status.\\
	
	The web application is designed using Angular.js, HTML5 and Cascading 
	Style Sheets (CSS). HTML is the accepted standard markup language 
	used when creating web pages, and will allow the web application to 
	run on a number of standard browsers. CSS will allow the presentation 
	of the web pages to be customised in a modular and reusable manner, 
	which is also stored separately from the actual content of the web 
	page. Angular.js is a framework which is used to determine the layout 
	and other functionality of web pages, with the potential of the 
	addition of Asynchronous JavaScript and XML (AJAX) functionality. 
	Angular.js is based on the Model-View-Control architecture.\\
	
	The second main access channel is the Particle device, which is 
	responsible for taking measurements from an attached appliance, and 
	sending those measurements on to the application server. The Particle 
	Photon communicates with the power meter using the an input pin on the Photon, 
	where the measuring device is responsible for taking current and 
	voltage readings from the appliance by using a current converter. The 
	Photon itself is interacted with using its firmware and the Particle 
	API. The Photon communicates with the application server using MQTT, 
	a lightweight addition to the TCP/IP protocol which allows for 
	reliable communication whilst requiring limited network bandwidth.\\
	
	When communicating with the Firebase service, the REsT architecture 
	is used. This allows communication over TCP/IP using HTTP verbs, 
	namely GET and POST in this case. GET is used to request data from 
	the Firebase database, whilst POST is used to send data to the 
	service. The REsT API allows the use of these commands, and controls 
	their usage.

\newpage

\section{Technologies}
	
	\subsection{Particle Photon}
	
	The Photon is a compact hardware board which contains a number of 
	built in components, such as a wi-fi module, light emitting diodes 
	(LEDs), an Analogue to Digital Converter (ADC), to name a few. The 
	Photon is interacted with using the Particle API, and its firmware is 
	coded using C++. The Photon is used for data storage, as well as data 
	transmission to the application server.\\
	
	\noindent
	There are no alternatives to this technology, as its usage is 
	specified by the client.
	
	\subsection{Particle.io}
	
	Particle.io is used in the both the firmware for the Particle Photon as well as to provide functionality for the web application. The Particle.io API is used to control Photon 
	functionality and hardware, and also includes a number of standard C++ classes. Additionally, Particle.io allows the usage of Particle cloud functionality, allowing functions and variables 
	to be accessed via the Internet.\\
	
	\noindent
	There are no alternatives to this technology, as its usage is specified by the client.
	
	
	\subsection{MQ Telemetry Transport}
	
	MQTT is a lightweight messaging protocol which is used on top of the 
	TCP/IP protocol. It is specially designed for usage in situations 
	where a small code footprint is required, and limited bandwidth is 
	available, which makes it suitable for usage within Internet of 
	Things (IoT) projects such as PowerCloud. Additionally, it provides 
	reliable communication, and allows for acknowledgement of successful 
	data reception from the receiver, and retransmission upon delivery 
	failure. MQTT functions using a publish/subscribe system.
	
	\subsection{Java Server}
	
	The application server is coded using Java. Java is chosen as it 
	comes with a well established knowledge-base, and is a mature 
	programming language which is readily supported. A Java server allows 
	for the usage of multiple threads, and also provides automated 
	garbage management services, amongst others. Java is also supported 
	by the Firebase service.
	
	\subsection{NodeJS}
	
	NodeJS is used to compile the web server. NodeJS is a run-time 
	environment used for server-side web applications. It provides an 
	event driven architecture that achieves asynchronous input and output 
	capabilities. It allows for a  persistent connection to be maintained 
	between the browser and the server, which is suitable for the 
	PowerCloud system. Any changes made on the data set can immediately 
	be propagated through the NodeJS server to the web application.
	
	\subsection{AngularJS}
	
	AngularJS will be used for the web application framework. It is a 
	powerful and flexible framework which provides modularization with 
	extensive module repositories, and allows real-time web page updates 
	without requiring a refresh by using AJAX. AngularJS uses two-way 
	binding and update the virtual Document Object Model (DOM), which 
	will be necessary within the web application. NodeJS is also well 
	supported by Firebase.
	
	\subsection{Firebase}
	
	Firebase is a persistence provider. It is a document store cloud 
	based database provider which stores data in a JavaScript Object 
	Notation format. It is tailored to provide scalability on request, 
	increasing the number of concurrent connections allowed with the 
	proportion to the size of the data set. A database built on 
	Firebase's infrastructure is accessible using their REsT API, as well 
	as through bindings for JavaScript frameworks, such as AngularJS. 
	Firebase provides real-time data synchronisation, with the database 
	immediately reflecting any changes across multiple clients. Firebase 
	servers are located in many major city centres around the world, 
	which allows optimal connection speeds , regardless of where it is 
	accessed from. These characteristics make Firebase an ideal choice 
	for PowerCloud, which focuses on real-time cloud based database 
	storage and access.
	
	\subsection{JavaScript Object Notation}
	
	JavaScript Object Notation (JSON) is a lightweight data-interchange 
	format which is easy to understand and manipulate. JSON is natively 
	supported across various technologies, and can often be used in its 
	raw form with no extra manipulation. This makes it ideal for use 
	within the PowerCloud system.
	
	\subsection{HighCharts}
	
	HighCharts is a JavaScript library which allows the addition of 
	interactive charts to a web application, with compatibility for most 
	modern browsers. It supports a large variety of graphical data 
	representations, and is free for non-commercial use. Being a pure 
	JavaScript library, it is set to work well with AngularJS and the 
	other chosen web application technologies. HighCharts provides 
	dynamic charts, which can be changed even after the chart has been 
	created, using event hooks. For these reasons, it is ideal for usage 
	within the PowerCloud system, due to its ease of use and 
	compatibility with previously chosen technologies. 

\end{document}          

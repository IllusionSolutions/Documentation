\documentclass{article}

%Packages
\usepackage{graphicx}
\usepackage{grffile}
\usepackage{float}

%Margins
\usepackage[
	margin=2cm,
	includefoot
	]{geometry}

%Images
\usepackage{graphicx}

\graphicspath{{images/}}

%Headers and Footers
\usepackage{fancyhdr}
\pagestyle{fancy}
\fancyhead{}
\fancyfoot{}
\fancyfoot[R]{\thepage}
\renewcommand{\headrulewidth}{0pt}
\renewcommand{\footrulewidth}{0pt}

% Title Page
\title{Architecture Specification}
\author
{  
	\\\\\\
	Stuart Andrews, 12153983 
	\\\\
	NAME, XXXXXXXX
	\\\\
	NAME, XXXXXXXX 
	\\\\
	Brandon Wardley, 29005150 
	\\\\\\\\\\\\\\
}

\begin{document}
\maketitle
\thispagestyle{empty}
\newpage
\tableofcontents

\newpage

\section{Architecture requirements}

This section contains a brief discussion of the software requirements of the PowerCloud system. It includes the architectural scope, the quality requirements, the integration requirements, access channel requirements, and architectural constraints.

	\subsection{Architectural scope}
	
	The software architecture used within the PowerCloud system must satisfy certain requirements brought forward by the client. These requirements need to be satisfied in order to ensure the system provides a satisfactory level of performance. The requirements are as follows:

	\begin{itemize}
		\item The system must provide a reliable persistence service which stores data in a readable format. This data must be easily maintainable.
		\item The system must be secure. Only authorised users should be able to access and use the system. This includes both human operators and Particle devices. Data sent within the system must be secure, and should not be accessible by outside parties.
		\item The system must provide reporting functionality. Stored data must be represented to the user in an understandable manner. This representation must be useful to the user, and provide additional understanding of the data as a whole, via the use of valid statistics.
		\item The system must provide reliability, such that the data sent and retrieved is both accurate and consistent. The system, and devices within the system, must recover gracefully if a failure does occur. If a part of the system fails, data should not be lost.
		\item The system must be able to cater to a large number of users, and no user should experience a noticeable lack in performance due to contention. This includes both human operators and Particle devices.
		\item The system should be easy to use, such that operators can easily access the required data, and that devices within the system be easy to configure and use.
		\item The system must provide compatibility, such that the data can be accessed from any computer which meets the minimum requirements.
	\end{itemize}
	
	\newpage

	\subsection{Quality requirements}
	
	The following quality requirements were established, and were found to be relevant to the PowerCloud system. These requirements should be quantifiable and testable, and are listed in order of priority.
	
	\begin{enumerate}
		\item Recoverability
			\begin{itemize}
				\item If a device goes offline, data should not lost for a given period, determined by the frequency at which measurements are taken.
				\item If a server were to cease operation, a replacement must be provided as soon as possible, in order to minimize impact on user operations.
			\end{itemize}
		\item Quality
			\begin{itemize}
				\item The server needs to ensure that data is in a valid format.
				\item The server needs to ensure that data is consistent, and if any deviations occur, the operators must be notified.
			\end{itemize}
		\item Reliability
			\begin{itemize}
				\item Data measured by the devices must be accurate, up to a given level specified by the operator.
				\item The server must ensure a 99.8\% uptime, outside of regular maintenance periods.
			\end{itemize}
		\item Security
			\begin{itemize}
				\item Device communication must be encrypted to avoid interception of sensitive data.
				\item Unauthorized operators and devices must not be able to access the system.
			\end{itemize}
		\item Usability
			\begin{itemize}
				\item The PowerCloud web interface must be intuitive and easy to use.
				\item Feedback must be provided to users after actions are performed, informing them as to the result of the action.
				\item If an error should occur, the user must be made aware of the error.
				\item Operation guidelines must be provided, to ensure the user can properly operate the system.
			\end{itemize}
		\item Scalability
			\begin{itemize}
				\item The server needs to handle up to 100 000 simultaneous connections from deployed devices.
			\end{itemize}
		\item Maintainability
			\begin{itemize}
				\item Any upgrades to the system must not change the contents in the database.
				\item A module within the system can be changed without affecting any of the other modules.
			\end{itemize}
	\end{enumerate}
	
	\newpage
	
	\subsection{Integration requirements}
	
	The first stage in the PowerCloud system uses the Particle device. The device is made up of two separate devices, one of which is responsible for taking readings from a desired appliance, and another which accepts these values and transfers them to an application server. The two devices communicate with one another using a Serial Peripheral Interface (SPI), which controls the transfer of data between the two devices.\\
	
	The second device, or the Photon, sends the readings to the application server by using MQ Telemetry Transport (MQTT), which is a publish-subscribe messaging protocol which is used on top of the TCP/IP protocol. Using MQTT, the Photon publishes data to the application server in a JavaScript Object Notation (JSON) format. The application server listens for any incoming MQTT messages, and processes those which it receives.\\
	
	The application server communicates with a FireBase persistence service using a Representational State Transfer (REsT) architecture, and as such communication uses the Hypertext Transfer Protocol (HTTP). This means that communication will rely on using HTTP verbs, such as POST and GET. The POST command is mainly used to send data to the FireBase service, such as when data needs to be stored in the FireBase database. The GET command is used to request data from the FireBase service, and is used when previously stored data needs to be retrieved from the database.\\
	
	In order for operators to access the data which has been stored on the FireBase database, a web interface must be provided. For this reason, a web server is used to connect to FireBase and retrieve any needed data from the database. The web server will communicate with FireBase in the same manner as the application server, using HTTP. The web server will then process this data, and display it to the operators via a web page.\\
	
	It is important to take note that each of these systems have certain quality requirements which they must provide for. Reliable communication must be provided, which will ensure that data is not lost in transmission, or otherwise altered. Security must be provided when transporting data over the public domain, to ensure that said data is not accessed by unauthorized entities.
	
	\newpage
	
	\subsection{Access channel requirements}
	
	Users, in the form of operators, will need to access the system to perform a number of tasks. The primary task will be to interact with and access data stored on the database. Additional tasks may include adding new devices to the system, or changing information relating to devices already on the system. In order to perform these tasks, a web application must be provided. This web application is provided by the web server. It is necessary that the web application must be easy to use and navigate, and must clearly display any and all required information. The web application must also provide feedback to users in a timely manner, such that they are made aware of the results of their actions, or of any alerts or notifications that may require their attention.\\
	
	The web application must be compatible with modern web browsers. As the web application will primarily be used to interact with the FireBase database, a working Internet connection will be required for it to operate. The web application will be based on the REsT architecture, and will communicate using the HTTP protocol.\\
	
	Another access channel that must be mentioned is the interaction between the device and the appliance which it is measuring. An electrical lead is used to attach the appliance to the device, as well as to power it from a power source. In order for the correct operations to occur, and to ensure the safety of both the device and nearby operators, it is required that the appliance plug and source both be correctly wired, as an incorrectly wired plug can cause damage to both the device and potential operators.
	
	\subsection{Architectural constraints}
	
	The only major constraint in the PowerCloud system is that it requires the use of the Particle device, through which readings are taken and transmitted. This enforces the use of the following:\\
	
	\noindent\textbf{Serial Peripheral Interface}\\
	Communication between the two devices which make up the measuring device makes use of the Serial Peripheral Interface protocol.\\
	
	\noindent\textbf{C++}\\
	The Particle device requires that the firmware be coded in C++, using the Particle API.\\

\newpage

\section{Architectural patterns and styles}

\newpage

\section{Architectural tactics and strategies}

\newpage

\section{Use of reference architectures and frameworks}

\newpage

\section{Access and integration channels}

\newpage

\section{Technologies}

\end{document}          

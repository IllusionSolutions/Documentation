\documentclass[a4paper,10pt]{article}

\usepackage{color}
\usepackage[utf8]{inputenc}

\title{Instruction manual for PowerCloud}
\author{Marc Antel\\ Brandon Wardley\\ Stuart Andrews\\ Mothusi Masibi}
\date{\today}

\begin{document}
		\maketitle
		\newpage
	\section{Compiling and Installing from Source Code}
		\subsection{Pre-requisites}
			\textbf{Software}
			Before proceeding please ensure you have the following software installed on your system:
			\begin{itemize}
				\item NodeJS and NPM
				\item Bower
				\item Gulp
				\item Particle Dev
				\item Java 8
				\item Maven
			\end{itemize}
			
			\textbf{Hardware}
			\begin{itemize}
				\item A Particle Photon.
				\item A computer with internet access which does not block incoming and outgoing connections on port 1883 and 80.
			\end{itemize}
			
		\subsection{Firmware}
			Compile and flash the photon with the following commands, please take note that if you run the application server on a local machine, retrieve the IP address and insert it into the firmware's code as shown below.\\

			Follow the instructions here in order to compile the firmware using Particle Dev.
			\\
			\begin{itemize}
				\item Open the firmware using Particle Dev.
				\item Click on "\textit{Compile using Partcicle Cloud}".
				\item Select the device to flash.
				\item Click on "\textit{flash device}".
			\end{itemize}
			
			\textit{If you run into errors, please see the troubleshooting guide at the end of this document.}
			
		\subsection{Application Server}
			Currently, the \textit{application-server} branch hosts the source code for the Application Server. It needs to be compiled using maven and run on a device which doesn't block incoming and outgoing connections on port 1883.\\
			
			\textbf{Compiling the Application Server}\\
			Download the sources from GitHub and compile:\\
			
			\begin{itemize}
				\item \textit{\$ mvn package}
			\end{itemize}		
				
			\textit{If you run into errors, please see the troubleshooting guide at the end of this document.}
			
			\newpage
	\section{Starting PowerCloud}
		\subsection{Firmware}
			Once the firmware has been compiled and the binary file produced. Flash the photon with the following command:
			
			\begin{itemize}
				\item \textit{particle serial flash Photon Name *.ino}
			\end{itemize}
			
		\subsection{Application Server}
			Once the application server has been compiled, change your directory to IllusionSolutions/test/, run the application server using the following command
			
			\begin{itemize}
				\item \textit{java -jar applicationServer-*.jar}
			\end{itemize}
			
		\subsection{Web Application}
			To start the PowerCloud Web application open the console.

			\begin{itemize}
			\item \textit{npm install}
			\item \textit{bower install}
			\item \textit{gulp serve}
			\end{itemize}
			
			The web application should now be available on \textbf{http://localhost:9000/}
			
			\newpage
	\section{Troubleshooting}
		\textbf{I am getting errors when trying to install through the console.}
			\begin{itemize}
			\item Try using sudo before the command. Example: sudo npm install particle-cli
			\end{itemize}
		
		\textbf{The Web Application won't run and is giving errors when executing the gulp serve command.} 
			\begin{itemize}
			\item Make sure you've run npm install and bower install.
			\item Make sure you're in the correct directory where the web application is located.
			\end{itemize}
\end{document}
\documentclass[a4paper,10pt]{article}

\usepackage{color}
\usepackage[utf8]{inputenc}

\title{Instruction manual for PowerCloud}
\author{Marc Antel\\ Brandon Wardley\\ Stuart Andrews\\ Mothusi Masibi}
\date{\today}

\begin{document}
		\maketitle
		\newpage
	\section{Compiling and Installing from Source Code}
		\subsection{Pre-requisites}
			\textbf{Software}
			Before proceeding please ensure you have the following software installed on your system:
			\begin{itemize}
				\item NodeJS and NPM
				\item Bower
				\item Gulp
				\item Particle Dev
				\item Java 8
			\end{itemize}
			
			\textbf{Hardware}
			\begin{itemize}
				\item A Particle Photon.
				\item A Computer with internet access which does not block incoming and outgoing connections on port 1883 and 80.
			\end{itemize}
		\subsection{Firmware}
			Compile and flash the photon with the following commands, please take note that if you run the application server on a local machine, retrieve the IP address and insert it into the firmware's code.\\
			
			\textbf{Compiling the Firmware}\\
			
			Follow the instructions here in order to compile the firmware using Particle Dev.
			
			\textit{If you run into errors, please see the troubleshooting guide at the end of this document.}
			
		\subsection{Application Server}
			Currently, the \textit{application-server} branch hosts the source code for the Application Server. It needs to be compiled using maven and run on a device which doesn't block incoming and outgoing connections on port 1883.\\
			
			\textbf{Compiling the Application Server}\\
			Download the sources from GitHub and compile:\\
			
			\textit{If you run into errors, please see the troubleshooting guide at the end of this document.}
		\subsection{Web Server}
			Open your console and redirect to the web-server files. Run npm install and bower install.
		
	\section{Starting PowerCloud}
		\subsection{Firmware}
			Once the firmware has been compiled and the binary file produced. Flash the photon with the following command:
		\subsection{Application Server}
		
		\subsection{Web Server}
			To start the PowerCloud Web-Server open the console, redirect to the Web-Server files and run the gulp serve command.
		
	\section{Troubleshooting}
		I am getting errors when trying to install through the console.
		-Try using sudo before the command. Example: sudo npm install particle-cli
		
		The Web-Server won't run and is giving errors when executing the gulp serve command.
		-Make sure you've run npm install and bower install. 
		-Make sure you're in the correct directory where the web-server is located.
\end{document}
